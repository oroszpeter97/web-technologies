\documentclass{article}

\usepackage[hungarian]{babel}
\usepackage{mathptmx}
\usepackage{titlesec}
\usepackage{ragged2e}
\usepackage{setspace}
\usepackage[margin=1.5cm]{geometry}
\usepackage[hidelinks]{hyperref}
\usepackage{graphicx}

\titleformat{\section}
  {\normalfont\fontsize{14}{16}\bfseries}{\thesection}{1em}{}

\titleformat{\subsection}
  {\normalfont\fontsize{12}{14}\bfseries}{\thesubsection}{1em}{}

\begin{document}
\onehalfspacing
\selectlanguage{hungarian}
\justifying 

\begin{titlepage}
    \centering
    \vspace*{0.25\textheight}
    {\fontsize{64}{32}\bfseries Jegyzőkönyv \par}
    \vspace{1.5em}
    {\fontsize{25}{64}\normalfont Web-technológiák \par}
    \vspace{1em}
    {\fontsize{25}{32}\normalfont Féléves feladat \par}
    \vspace{1em}
    {\fontsize{25}{32}\normalfont Recept Weboldal \par}
    \vspace*{0.20\textheight}

    \begin{flushright}
        \begin{tabular}{@{}l@{}}
            {\fontsize{14}{14}\normalfont Készítette: Orosz Péter} \\
            {\fontsize{14}{14}\normalfont Neptun kód: WO02D7} \\
            {\fontsize{14}{14}\normalfont Dátum: 2025. December} \\
        \end{tabular}
    \end{flushright}
\end{titlepage}

\tableofcontents
\newpage

\section{Bevezetés}
Ez a jegyzőkönyv röviden bemutatja a projekt célját és felépítését. A projekt egy egyszerű receptoldal, amely lehetővé teszi receptek megtekintését, hozzáadását és eltávolítását.

\section{Projekt leírása}
\subsection{Cél}
A cél egy szemantikusan jól felépített, hozzáférhető és egyszerűen használható receptkezelő weboldal készítése, amely oktatási céllal készült a Web-technológiák kurzushoz.

\subsection{Funkciók}
\begin{itemize}
  \item Receptek megtekintése (lista, részletes nézet).
  \item Új recept beküldése (cím, hozzávalók, leírás).
  \item Receptek eltávolítása.
  \item Beágyazott oktatóvideó a főoldalon.
\end{itemize}

\subsection{Mappa- és fájlszerkezet}
Rövid áttekintés a fontos fájlokról:
\begin{itemize}
  \item index.html — kezdőoldal
  \item recipes.html — receptlista
  \item add-recipes.html — recept hozzáadása
  \item remove-recipe.html — recept eltávolítása
  \item styles/styles.css — alap stílusok
  \item resources/ — képek és egyéb erőforrások
\end{itemize}

\section{Használati útmutató}
A weboldal használata egyszerű:
\begin{enumerate}
  \item Indítsa el a szervert a projekt gyökérkönyvtárában:
        \texttt{node server.js}
  \item A "View Recipes" oldalon böngészheti a meglévő recepteket.
  \item Az "Add Recipe" oldalon töltsön ki minden szükséges mezőt, majd küldje be a receptet.
  \item A "Remove Recipe" oldalon törölheti a nem kívánt bejegyzéseket.
\end{enumerate}

\section{Műszaki megjegyzések}
\subsection{Elérhetőség és szemantika}
A HTML szemantikai elemeket (header, nav, main, article, aside, footer) használjuk a jobb strukturáltságért és akadálymentességért. A képekhez és videókhoz alternatív leírásokat és feliratokat adtunk meg ahol szükséges.

\subsection{Reszponzivitás}
A beágyazott videó és a képek rugalmasan skálázódnak, így a tartalom különböző képernyőméreteken is olvasható marad.

\subsection{Fejlesztési megjegyzések}
A projekt egyszerű, statikus fájlokra épül; további funkciók (pl. adatbázis, szerveroldali feldolgozás) szükség esetén hozzáadhatók.

\section{Fontos kódrészletek}
Az alábbi rövid kódrészletek a projekt működéséhez gyakran használt mintákat mutatják: egyszerű Node/Express szerver, kliens oldali recept beküldés és törlés fetch segítségével.

\subsection{server.js — egyszerű Express szerver és API}
\begin{verbatim}
const express = require('express');
const app = express();

app.use(express.json());
app.use(express.static('.')); // statikus fájlok kiszolgálása a projekt gyökérből

let recipes = []; // egyszerű in-memory tárolás

app.get('/api/recipes', (req, res) => {
  res.json(recipes);
});

app.post('/api/recipes', (req, res) => {
  const recipe = req.body;
  recipe.id = Date.now();
  recipes.push(recipe);
  res.status(201).json(recipe);
});

app.delete('/api/recipes/:id', (req, res) => {
  const id = Number(req.params.id);
  recipes = recipes.filter(r => r.id !== id);
  res.status(204).end();
});

app.listen(3000, () => console.log('Server running at http://localhost:3000'));
\end{verbatim}

\subsection{Kliens: recept beküldése (POST)}
\begin{verbatim}
const form = document.querySelector('#addRecipeForm');
form.addEventListener('submit', async (e) => {
  e.preventDefault();
  const data = {
    title: form.title.value,
    ingredients: form.ingredients.value,
    instructions: form.instructions.value
  };
  const res = await fetch('/api/recipes', {
    method: 'POST',
    headers: { 'Content-Type': 'application/json' },
    body: JSON.stringify(data)
  });
  if (res.ok) window.location.href = '/recipes.html';
});
\end{verbatim}

\subsection{Kliens: recept törlése (DELETE)}
\begin{verbatim}
async function removeRecipe(id) {
  const res = await fetch(`/api/recipes/${id}`, { method: 'DELETE' });
  if (res.ok) {
    const el = document.querySelector(`#recipe-${id}`);
    if (el) el.remove();
  }
}
\end{verbatim}

\subsection{Kliens: receptek betöltése (GET)}
\begin{verbatim}
async function loadRecipes() {
  const res = await fetch('/api/recipes');
  const recipes = await res.json();
  // egyszerű render például:
  const list = document.querySelector('#recipesList');
  list.innerHTML = recipes.map(r => `
    <li id="recipe-${r.id}">
      <h3>${r.title}</h3>
      <button onclick="removeRecipe(${r.id})">Törlés</button>
    </li>
  `).join('');
}
document.addEventListener('DOMContentLoaded', loadRecipes);
\end{verbatim}

\paragraph{Miért emeltem ki ezeket a kódrészleteket?}
Az itt bemutatott példák lefedik az alkalmazás alapvető működését: a szerver (server.js) szolgálja ki a statikus fájlokat és biztosítja az API végpontokat, a kliensoldali POST művelet létrehozza az új recepteket, a DELETE kezeli a törlést és az UI frissítését, míg a GET felel a receptek lekéréséért és megjelenítéséért. Ezek megértése elengedhetetlen a hibakereséshez, a funkcionalitás bővítéséhez és a biztonsági/validációs rétegek hozzáadásához. A példák egyszerűsítettek, de könnyen kiterjeszthetők tartós tárolásra (adatbázis), autentikációra és részletes hibakezelésre.

\end{document}